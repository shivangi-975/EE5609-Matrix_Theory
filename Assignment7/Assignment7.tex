\documentclass[journal,12pt,twocolumn]{IEEEtran}

\usepackage{setspace}
\usepackage{gensymb}

\singlespacing


\usepackage[cmex10]{amsmath}

\usepackage{amsthm}

\usepackage{mathrsfs}
\usepackage{txfonts}
\usepackage{stfloats}
\usepackage{bm}
\usepackage{cite}
\usepackage{cases}
\usepackage{subfig}

\usepackage{longtable}
\usepackage{multirow}

\usepackage{enumitem}
\usepackage{mathtools}
\usepackage{steinmetz}
\usepackage{tikz}
\usepackage{circuitikz}
\usepackage{verbatim}
\usepackage{tfrupee}
\usepackage[breaklinks=true]{hyperref}

\usepackage{tkz-euclide}

\usetikzlibrary{calc,math}
\usepackage{listings}
    \usepackage{color}                                            %%
    \usepackage{array}                                            %%
    \usepackage{longtable}                                        %%
    \usepackage{calc}                                             %%
    \usepackage{multirow}                                         %%
    \usepackage{hhline}                                           %%
    \usepackage{ifthen}                                           %%
    \usepackage{lscape}     
\usepackage{multicol}
\usepackage{chngcntr}

\DeclareMathOperator*{\Res}{Res}

\renewcommand\thesection{\arabic{section}}
\renewcommand\thesubsection{\thesection.\arabic{subsection}}
\renewcommand\thesubsubsection{\thesubsection.\arabic{subsubsection}}

\renewcommand\thesectiondis{\arabic{section}}
\renewcommand\thesubsectiondis{\thesectiondis.\arabic{subsection}}
\renewcommand\thesubsubsectiondis{\thesubsectiondis.\arabic{subsubsection}}


\hyphenation{op-tical net-works semi-conduc-tor}
\def\inputGnumericTable{}                                 %%

\lstset{
%language=C,
frame=single, 
breaklines=true,
columns=fullflexible
}
\begin{document}


\newtheorem{theorem}{Theorem}[section]
\newtheorem{problem}{Problem}
\newtheorem{proposition}{Proposition}[section]
\newtheorem{lemma}{Lemma}[section]
\newtheorem{corollary}[theorem]{Corollary}
\newtheorem{example}{Example}[section]
\newtheorem{definition}[problem]{Definition}

\newcommand{\BEQA}{\begin{eqnarray}}
\newcommand{\EEQA}{\end{eqnarray}}
\newcommand{\define}{\stackrel{\triangle}{=}}
\bibliographystyle{IEEEtran}
\providecommand{\mbf}{\mathbf}
\providecommand{\pr}[1]{\ensuremath{\Pr\left(#1\right)}}
\providecommand{\qfunc}[1]{\ensuremath{Q\left(#1\right)}}
\providecommand{\sbrak}[1]{\ensuremath{{}\left[#1\right]}}
\providecommand{\lsbrak}[1]{\ensuremath{{}\left[#1\right.}}
\providecommand{\rsbrak}[1]{\ensuremath{{}\left.#1\right]}}
\providecommand{\brak}[1]{\ensuremath{\left(#1\right)}}
\providecommand{\lbrak}[1]{\ensuremath{\left(#1\right.}}
\providecommand{\rbrak}[1]{\ensuremath{\left.#1\right)}}
\providecommand{\cbrak}[1]{\ensuremath{\left\{#1\right\}}}
\providecommand{\lcbrak}[1]{\ensuremath{\left\{#1\right.}}
\providecommand{\rcbrak}[1]{\ensuremath{\left.#1\right\}}}
\theoremstyle{remark}
\newtheorem{rem}{Remark}
\newcommand{\sgn}{\mathop{\mathrm{sgn}}}
\providecommand{\abs}[1]{\left\vert#1\right\vert}
\providecommand{\res}[1]{\Res\displaylimits_{#1}} 
\providecommand{\norm}[1]{\left\lVert#1\right\rVert}
%\providecommand{\norm}[1]{\lVert#1\rVert}
\providecommand{\mtx}[1]{\mathbf{#1}}
\providecommand{\mean}[1]{E\left[ #1 \right]}
\providecommand{\fourier}{\overset{\mathcal{F}}{ \rightleftharpoons}}
%\providecommand{\hilbert}{\overset{\mathcal{H}}{ \rightleftharpoons}}
\providecommand{\system}{\overset{\mathcal{H}}{ \longleftrightarrow}}
	%\newcommand{\solution}[2]{\textbf{Solution:}{#1}}
\newcommand{\solution}{\noindent \textbf{Solution: }}
\newcommand{\cosec}{\,\text{cosec}\,}
\providecommand{\dec}[2]{\ensuremath{\overset{#1}{\underset{#2}{\gtrless}}}}
\newcommand{\myvec}[1]{\ensuremath{\begin{pmatrix}#1\end{pmatrix}}}
\newcommand{\mydet}[1]{\ensuremath{\begin{vmatrix}#1\end{vmatrix}}}
\numberwithin{equation}{subsection}
\makeatletter
\@addtoreset{figure}{problem}
\makeatother
\let\StandardTheFigure\thefigure
\let\vec\mathbf
\renewcommand{\thefigure}{\theproblem}
\def\putbox#1#2#3{\makebox[0in][l]{\makebox[#1][l]{}\raisebox{\baselineskip}[0in][0in]{\raisebox{#2}[0in][0in]{#3}}}}
     \def\rightbox#1{\makebox[0in][r]{#1}}
     \def\centbox#1{\makebox[0in]{#1}}
     \def\topbox#1{\raisebox{-\baselineskip}[0in][0in]{#1}}
     \def\midbox#1{\raisebox{-0.5\baselineskip}[0in][0in]{#1}}
\vspace{3cm}
\title{Assignment 7}
\author{Shivangi Parashar}
\maketitle
\newpage
\bigskip
\renewcommand{\thefigure}{\theenumi}
\renewcommand{\thetable}{\theenumi}
\begin{abstract}
This document explains the method of performing QR decomposition on a 2$\times$2 matrix.
\end{abstract}
Download all python codes from 
\begin{lstlisting}
https://github.com/shivangi-975/EE5609-Matrix_Theory/tree/master/Assignment7/codes
\end{lstlisting}
%
and latex-tikz codes from 
%
\begin{lstlisting}
https://github.com/shivangi-975/EE5609-Matrix_Theory/edit/master/Assignment7/Assignment7.tex
\end{lstlisting}
%

\section{Problem}
Find the QR decomposition  on a given 2$\times$2 matrix. 
\begin{align}
    \myvec{2&1\\1&-2}
\end{align}
\section{Solution}
The QR decomposition  of a matrix is a decomposition of the matrix into an orthogonal matrix and an upper triangular matrix.
QR decomposition of a square matrix is given by,
\begin{align}
    \vec{A} = \vec{Q}\vec{R}
\end{align}
Here  $\vec{Q}$ is an orthogonal matrix and $\vec{R}$ is an upper triangular matrix.\\
\\
Given matrix,
\begin{align}
    \vec{A} = 
    \myvec{2&1\\
    1&-2} \label{eq:eq1}
\end{align}
The column vectors of the matrix is given by,
\begin{align}
    \vec{a}=\myvec{2\\1} \quad \vec{b}=\myvec{1\\-2} \label{eq:eq2}
\end{align}
Equation \eqref{eq:eq1} can be written in \vec{Q}\vec{R} form as:
\begin{align}
    \vec{Q}\vec{R} = \myvec{\vec{q_1}&\vec{q_2}} \myvec{u_1&u_3\\0&u_2}  \label{eq:eq7}
\end{align}
Now, 
\begin{align}
u_1 &= \norm{\vec{a}} = \sqrt{1^2+2^2} = \sqrt{5}\label{eq:eq3}\\
\vec{q_1} &= \frac{\vec{a}}{u_1} = \myvec{\frac{2}{\sqrt{5}}\\ \frac{1}{\sqrt{5}}}\\
u_3 &= \frac{\vec{q_1}^T\vec{b}}{\norm{\vec{q_1}}^2} = \myvec{\frac{2}{\sqrt{5}}& \frac{1}{\sqrt{5}}}\myvec{1\\-2} = 0\\
\vec{q_2} &= \frac{\vec{b} - u_3 \vec{q_1}}{\norm{\vec{b} - u_3 \vec{q_1}}} = \myvec{\frac{1}{\sqrt{5}}\\ -\frac{2}{\sqrt{5}}}\\
u_2 &= {\vec{q_2}^T\vec{b}} = \myvec{\frac{1}{\sqrt{5}}&-\frac{2}{\sqrt{5}}}\myvec{1\\-2}=\sqrt{5}\label{eq:eq4}
\end{align}
Substituting equation \eqref{eq:eq3} to \eqref{eq:eq4} in \eqref{eq:eq7},to
the QR Decomposition of the given matrix is:
\begin{align}
    \myvec{2&1\\1&-2} = \myvec{\frac{2}{\sqrt{5}}&\frac{1}{\sqrt{5}}\\\frac{1}{\sqrt{5}}&-\frac{2}{\sqrt{5}}}\myvec{\sqrt{5}&0\\0&{\sqrt{5}}}
\end{align}
\\\end{document}

 


