\documentclass[journal,12pt,twocolumn]{IEEEtran}

\usepackage{setspace}
\usepackage{gensymb}

\singlespacing


\usepackage[cmex10]{amsmath}

\usepackage{amsthm}

\usepackage{mathrsfs}
\usepackage{txfonts}
\usepackage{stfloats}
\usepackage{bm}
\usepackage{cite}
\usepackage{cases}
\usepackage{subfig}

\usepackage{longtable}
\usepackage{multirow}

\usepackage{enumitem}
\usepackage{mathtools}
\usepackage{steinmetz}
\usepackage{tikz}
\usepackage{circuitikz}
\usepackage{verbatim}
\usepackage{tfrupee}
\usepackage[breaklinks=true]{hyperref}
\usepackage{graphicx}
\usepackage{tkz-euclide}

\usetikzlibrary{calc,math}
\usepackage{listings}
    \usepackage{color}                                            %%
    \usepackage{array}                                            %%
    \usepackage{longtable}                                        %%
    \usepackage{calc}                                             %%
    \usepackage{multirow}                                         %%
    \usepackage{hhline}                                           %%
    \usepackage{ifthen}                                           %%
    \usepackage{lscape}     
\usepackage{multicol}
\usepackage{chngcntr}

\DeclareMathOperator*{\Res}{Res}

\renewcommand\thesection{\arabic{section}}
\renewcommand\thesubsection{\thesection.\arabic{subsection}}
\renewcommand\thesubsubsection{\thesubsection.\arabic{subsubsection}}

\renewcommand\thesectiondis{\arabic{section}}
\renewcommand\thesubsectiondis{\thesectiondis.\arabic{subsection}}
\renewcommand\thesubsubsectiondis{\thesubsectiondis.\arabic{subsubsection}}


\hyphenation{op-tical net-works semi-conduc-tor}
\def\inputGnumericTable{}                                 %%

\lstset{
%language=C,
frame=single, 
breaklines=true,
columns=fullflexible
}
\begin{document}


\newtheorem{theorem}{Theorem}[section]
\newtheorem{problem}{Problem}
\newtheorem{proposition}{Proposition}[section]
\newtheorem{lemma}{Lemma}[section]
\newtheorem{corollary}[theorem]{Corollary}
\newtheorem{example}{Example}[section]
\newtheorem{definition}[problem]{Definition}

\newcommand{\BEQA}{\begin{eqnarray}}
\newcommand{\EEQA}{\end{eqnarray}}
\newcommand{\define}{\stackrel{\triangle}{=}}
\bibliographystyle{IEEEtran}
\providecommand{\mbf}{\mathbf}
\providecommand{\pr}[1]{\ensuremath{\Pr\left(#1\right)}}
\providecommand{\qfunc}[1]{\ensuremath{Q\left(#1\right)}}
\providecommand{\sbrak}[1]{\ensuremath{{}\left[#1\right]}}
\providecommand{\lsbrak}[1]{\ensuremath{{}\left[#1\right.}}
\providecommand{\rsbrak}[1]{\ensuremath{{}\left.#1\right]}}
\providecommand{\brak}[1]{\ensuremath{\left(#1\right)}}
\providecommand{\lbrak}[1]{\ensuremath{\left(#1\right.}}
\providecommand{\rbrak}[1]{\ensuremath{\left.#1\right)}}
\providecommand{\cbrak}[1]{\ensuremath{\left\{#1\right\}}}
\providecommand{\lcbrak}[1]{\ensuremath{\left\{#1\right.}}
\providecommand{\rcbrak}[1]{\ensuremath{\left.#1\right\}}}
\theoremstyle{remark}
\newtheorem{rem}{Remark}
\newcommand{\sgn}{\mathop{\mathrm{sgn}}}
\providecommand{\abs}[1]{\left\vert#1\right\vert}
\providecommand{\res}[1]{\Res\displaylimits_{#1}} 
\providecommand{\norm}[1]{\left\lVert#1\right\rVert}
%\providecommand{\norm}[1]{\lVert#1\rVert}
\providecommand{\mtx}[1]{\mathbf{#1}}
\providecommand{\mean}[1]{E\left[ #1 \right]}
\providecommand{\fourier}{\overset{\mathcal{F}}{ \rightleftharpoons}}
%\providecommand{\hilbert}{\overset{\mathcal{H}}{ \rightleftharpoons}}
\providecommand{\system}{\overset{\mathcal{H}}{ \longleftrightarrow}}
	%\newcommand{\solution}[2]{\textbf{Solution:}{#1}}
\newcommand{\solution}{\noindent \textbf{Solution: }}
\newcommand{\cosec}{\,\text{cosec}\,}
\providecommand{\dec}[2]{\ensuremath{\overset{#1}{\underset{#2}{\gtrless}}}}
\newcommand{\myvec}[1]{\ensuremath{\begin{pmatrix}#1\end{pmatrix}}}
\newcommand{\mydet}[1]{\ensuremath{\begin{vmatrix}#1\end{vmatrix}}}
\numberwithin{equation}{subsection}
\makeatletter
\@addtoreset{figure}{problem}
\makeatother
\let\StandardTheFigure\thefigure
\let\vec\mathbf
\renewcommand{\thefigure}{\theproblem}
\def\putbox#1#2#3{\makebox[0in][l]{\makebox[#1][l]{}\raisebox{\baselineskip}[0in][0in]{\raisebox{#2}[0in][0in]{#3}}}}
     \def\rightbox#1{\makebox[0in][r]{#1}}
     \def\centbox#1{\makebox[0in]{#1}}
     \def\topbox#1{\raisebox{-\baselineskip}[0in][0in]{#1}}
     \def\midbox#1{\raisebox{-0.5\baselineskip}[0in][0in]{#1}}
\vspace{3cm}
\title{Assignment 19}
\author{Shivangi Parashar\\AI20MTECH14012}
\maketitle
\newpage
\bigskip
\renewcommand{\thefigure}{\theenumi}
\renewcommand{\thetable}{\theenumi}
Download latex-tikz codes from 
%
\begin{lstlisting}
https://github.com/shivangi-975/EE5609-Matrix_Theory/blob/master/Assignment19/Assignment_19.tex
\end{lstlisting}
%
 
\section{QUESTION}
Let A be a $3\times 3$  matrix  with real entries.Identify  the correct statements.

1.A  is necessarily diagonalizable over $\vec{R}$

2.If A has distinct real  eigen values than  it is diagonalizable over$\vec{R}$

3.If A has distinct eigen values than  it is diagonalizable over $\vec{C}$

4.If all eigen values are non zero than it is diagonalizable over $\vec{C}$
%
\section{Solution}
\begin{table*}[!t]
    \centering
    \begin{tabular}{|l|l|}
    \hline
    & \\
Statement 1. & A  is necessarily diagonalizable over $\vec{R}$\\
\hline
& \\
False statement& Matrix A is diagonalizable if and only if there is a basis of $\vec{R}^3 $consisting of\\
& eigenvectors of A.\\
Example:&Consider a matrix\\&\parbox{12cm}{\begin{align}
 \myvec{
1 &1 &0\\
0&1&1\\
0&0&4}\end{align}}\\
&Eigen values are:\\
&\parbox{12cm}{\begin{align}
 \myvec{
1 -\lambda &1 &0\\
0&1-\lambda&1\\
0&0&4-\lambda}=0.
\implies\lambda_1=1,\lambda_2=4\end{align}}\\
&\parbox{12cm}{\begin{align}
  \lambda_1=1\text { has eigen vector}
 \myvec{1\\0\\0} \text{and} 
  \lambda_2=4 \text{ has eigen vector}
\myvec{1\\3\\9}
\end{align}}\\
 & We have found only two linearly independent eigenvectors for A,not diagonalisable
\\
\hline
& \\
Statement 2. & If A has distinct real  eigen values
 than  it is diagonalizable over$\vec{R}$\\
\hline
&\\
True statement& A has n distinct eigen values $\implies$ n linearly independent eigenvectors\\
 & $\implies$   diagonalizable.\\
Proof  1:& \textbf{Distinct eigen values implies distinct eigen vectors}\\&Consider 3  eigen vectors $\vec{v}$,$\vec{w}$ and $\vec{u}$ with eigen values $\lambda$,$\mu$,$\nu$ respectively.\\
& such that $\lambda\neq\mu\neq\nu$\\
 &\parbox{12cm}{\begin{align}
    \alpha(\vec{v})+\beta(\vec{w})+\gamma(\vec{u})=0\label{eq2}\\
     \alpha A(\vec{v})+\beta A(\vec{w})+\gamma A (\vec{u})=0\\
     \alpha \lambda\vec{v}+\beta\mu\vec{w}+\gamma\nu\vec{u}=0\label{eq3}
     \end{align}}\\
     & Multiplying $\eqref{eq2}$with -$\lambda$ and subtracting from $\eqref{eq3}$ we have,\\
   & \parbox{12cm}{\begin{align}  
  \beta(\mu-\lambda)\vec{w}+\gamma(\nu-\lambda)\vec{u}=0 \label{eq1}
  \end{align}}\\
  &  From equation$\eqref{eq1}$ we have, $\beta=\gamma=0$\\
  & substituting $\beta=\gamma=0$ in  equation $\eqref{eq2}$we have,$\alpha=\beta=\gamma=0$\\
  & \textbf{which proves that vectors are linearly independent}.\\
  Proof 2:
 & \textbf{If vectors are linearly independent than matrix is diagonalizable}\\
 & If\myvec{
\vec{p_1} &\vec{p_2}&\cdots&\vec{p_n} 
}are n independent eigen vectors then,
 $A\vec{p_1}=\lambda \vec{p_1},\cdots ,A\vec{p_n}=\lambda \vec{p_n}$\\
&\parbox{12cm}{\begin{align}{D}=\myvec{\lambda_1&0&\cdots&0\\
0&\lambda_2&\cdots&0\\
\vdots&\vdots&\vdots&\vdots\\
0&0&\cdots&\lambda_n}
 \and{P}=\myvec{
\vec{P_1}& \vec{P_2}&\cdots& \vec{P_n}
}\end{align}}\\
& Now, $A\vec{P_i}=\lambda_i\vec{P_i}$ $\implies$ ${A}{P}={P}{D}$\\
& so,${P^{-1}}{AP}={D}$ is a diagonal matrix.\\
\hline
    \end{tabular}
    \caption*{TABLE 1:Solution}
\end{table*}
\begin{table*}[!t]
    \centering
    \begin{tabular}{|l|l|}
    \hline
& \\
Statement 3. &If all eigen values are non zero than it is diagonalizable over $\vec{C}$\\
\hline
&\\
True statement& If A is an $N \times N$ complex matrix with n distinct eigenvalues, then any set of\\
& n corresponding eigenvectors form a basis for $\vec{C}^n$\\ .
&\\
Proof: &It is sufficient to prove that the set of eigenvectors is linearly independent \\
&which is proved in statement 2.\\
 Example:&\parbox{12cm}{\begin{align}A=\myvec{
4& 0 &-2\\
2& 5 &4\\
0& 0 &5
}\end{align}}\\
& Eigen values of A are:\\
& \parbox{12cm}{\begin{align}\lambda_1=2,\lambda_2=3 , \lambda_3=6\end{align}}\\
& Eigen vectors are:\\&\parbox{12cm}{\begin{align}x_1=\myvec{
-1\\1\\0
},
x_ 2=\myvec{
1\\1\\1
},
x_3=\myvec{
-1\\ - 1\\ 2
}\end{align}}
\\
 & Matrix A is diagonalizable because there is a basis of $\vec{C}^3 $consisting of\\
& eigenvectors of A.\\
\hline
& \\
Statement 4. & If all eigen values are non zero than it is diagonalizable over $\vec{C}$\\
\hline
& \\
False statement & 
counter example same as statement 1 ex. eigen values are non zero but \\&it is not diagonalizable. \\
\hline
    \end{tabular}
    \caption*{TABLE 2:Solution}
\end{table*}
\end{document}


   
